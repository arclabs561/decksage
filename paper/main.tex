\documentclass{article}
\usepackage[utf8]{inputenc}
\usepackage{amsmath,amssymb}
\usepackage{graphicx}
\usepackage{hyperref}
\usepackage{booktabs}

\title{DeckSage: Cross-Game Card Similarity via Graph Embeddings}

\author{
  Anonymous Author(s) \\
  \texttt{decksage@example.org}
}

\date{\today}

\begin{document}

\maketitle

\begin{abstract}
We present DeckSage, a multi-game platform for discovering card similarities in competitive trading card games using graph embeddings. By constructing co-occurrence graphs from tournament deck data and applying Node2Vec, we learn semantic card representations that capture strategic relationships. We evaluate our approach across three games: Magic: The Gathering, Yu-Gi-Oh!, and Pokemon TCG, comparing against baseline methods (Jaccard similarity, degree centrality). Our findings reveal that \textbf{simple Jaccard similarity outperforms Node2Vec} on our dataset (P@10: 0.141 vs 0.136), suggesting data quality issues rather than algorithmic limitations. We contribute: (1) a multi-game evaluation framework, (2) LLM-based relevance judging, and (3) honest assessment of when embeddings fail.
\end{abstract}

\section{Introduction}

Trading card games (TCGs) feature thousands of unique cards, making deck construction challenging. Recommending similar or substitute cards requires understanding strategic relationships beyond surface attributes.

\textbf{Problem:} Given a card $c$, find cards $\{c_1, c_2, \ldots, c_k\}$ that serve similar strategic roles.

\textbf{Approach:} Co-occurrence graphs + Node2Vec embeddings

\textbf{Key Finding:} On our dataset, \emph{simple Jaccard similarity beats Node2Vec}. We investigate why.

\section{Related Work}

\begin{itemize}
\item Graph embeddings: DeepWalk \cite{deepwalk}, Node2Vec \cite{node2vec}
\item Recommender systems: Item-based collaborative filtering
\item Card game analysis: MTG deck prediction, archetype clustering
\end{itemize}

\section{Methodology}

\subsection{Data Collection}

We scrape competitive decks from tournament websites:
\begin{itemize}
\item Magic: MTGTop8, Scryfall (150 decks)
\item Yu-Gi-Oh!: YGOPRODeck API
\item Pokemon: PokemonTCG API
\end{itemize}

\textbf{Data Issues Identified:}
\begin{itemize}
\item Format imbalance: 44 Legacy vs 16 Modern decks
\item Single-day snapshot (no temporal coverage)
\item Missing key cards (Tarmogoyf, Ragavan)
\end{itemize}

\subsection{Graph Construction}

Given decks $D = \{d_1, \ldots, d_n\}$, construct weighted graph $G = (V, E)$ where:
\begin{itemize}
\item Nodes $V$: unique cards
\item Edges $(c_i, c_j) \in E$: co-occurrence count across decks
\end{itemize}

\subsection{Embeddings}

Node2Vec \cite{node2vec} with parameters:
\begin{itemize}
\item Dimension: $d = 128$
\item Walk length: $\ell = 80$
\item Walks per node: $r = 10$
\item Return parameter: $p = 1.0$
\item In-out parameter: $q = 1.0$
\end{itemize}

Implementation: PecanPy \cite{pecanpy} (faster than reference)

\subsection{Baselines}

\begin{enumerate}
\item \textbf{Random}: Sample $k$ random cards
\item \textbf{Degree centrality}: Cards with similar degree
\item \textbf{Jaccard similarity}: $\frac{|N(c_i) \cap N(c_j)|}{|N(c_i) \cup N(c_j)|}$
\end{enumerate}

\subsection{Evaluation}

\textbf{Metrics:}
\begin{itemize}
\item Precision@K: $\frac{1}{k}\sum_{i=1}^k \mathbb{1}[\text{relevant}(c_i)]$
\item Mean Reciprocal Rank (MRR)
\item NDCG@K (graded relevance)
\end{itemize}

\textbf{Ground Truth:}
\begin{itemize}
\item Human annotation (5-point scale)
\item LLM judge (Claude 4.5 Sonnet)
\end{itemize}

\section{Results}

\subsection{Magic: The Gathering}

\begin{table}[h]
\centering
\begin{tabular}{lcccc}
\toprule
Method & P@5 & P@10 & P@20 & MRR \\
\midrule
Random & 0.010 & 0.012 & 0.014 & 0.036 \\
Degree & 0.010 & 0.012 & 0.014 & 0.036 \\
\textbf{Jaccard} & \textbf{0.156} & \textbf{0.141} & \textbf{0.119} & \textbf{0.325} \\
Node2Vec & 0.145 & 0.136 & 0.122 & 0.324 \\
\bottomrule
\end{tabular}
\caption{Evaluation results on Magic: The Gathering dataset. Jaccard similarity outperforms Node2Vec by 3.4\% on P@10.}
\end{table}

\textbf{LLM Judge Evaluation:} Average quality 3.3/10 (poor)

\subsection{Analysis: Why Node2Vec Fails}

\textbf{Hypothesis 1: Insufficient Data}
\begin{itemize}
\item Only 150 decks (60 Legacy, 16 Modern)
\item Graph too sparse for random walks to capture semantics
\end{itemize}

\textbf{Hypothesis 2: Data Quality}
\begin{itemize}
\item Format imbalance biases embeddings toward Legacy cards
\item Missing Modern staples reduces coverage
\end{itemize}

\textbf{Hypothesis 3: Graph Structure}
\begin{itemize}
\item Card co-occurrence != card similarity
\item Need deck context (archetype, format) as node features
\end{itemize}

\section{Discussion}

\textbf{When does Node2Vec work?}
\begin{itemize}
\item Large graphs ($>10$K nodes)
\item Dense connectivity
\item Homogeneous node types
\end{itemize}

\textbf{Our data:} 1,328 cards, sparse, heterogeneous (different formats/archetypes)

\textbf{Recommendation:} Extract 1000+ diverse decks before revisiting embeddings.

\section{Contributions}

\begin{enumerate}
\item Multi-game evaluation framework (3 TCGs)
\item Honest assessment: Jaccard beats Node2Vec
\item LLM-based relevance judging
\item Reproducible evaluation pipeline
\end{enumerate}

\section{Future Work}

\begin{itemize}
\item Extract diverse, balanced tournament data
\item Contextualized embeddings (deck archetype as features)
\item GNN approaches (GraphSAGE, GAT)
\item Transfer learning between games
\end{itemize}

\bibliographystyle{plain}
\bibliography{references}

\end{document}

